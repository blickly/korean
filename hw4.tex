\documentclass[11pt]{article}
\usepackage{kotex}
\usepackage{setspace}
\usepackage{fullpage}
\renewcommand \thesection{\Roman{section}}

\title{한국어 102: 숙제 \#4}
\author{벤}
\date{2010년 3월 14일}

\begin{document}
\maketitle
\thispagestyle{empty}
\pagestyle{empty}

\section{표현}
\begin{enumerate}
\item 성공을 거두다 \\
   빌 게이츠는 대학을 중퇴했어도, 회사로 성공을 거뒀다.

\item 의견을 내다/제시하다 \\
  연구를 잘해도 의견을 제시하러 학회에 안 가면, 아무도 모를 거다.

\item 의견을/뜻을/요구를/요청을 받아들이다/따르다/감안하다/반영하다
\begin{enumerate}
\item 손님의 요구를 따라서 누드스시가 새벽 3시까지 배달한다.
\item 토론할 때 상대방의 의견을 감안할수록 답을 잘 생각한다. 
\end{enumerate}
\item (불)가능한 일로 여기다 \\
  1961년에 인간이 달까지 가는 것은 불가능한 일로 여겼지만, 케네디의 연설 든 다음에 의견을 바꿨다.
\item 맛을 내다 \\
  파스타를 삶을 때 밀맛을 내려면, 소금을 넣어야 한다.
\item 가격을 올리다/내리다 \\
  성 밸런타인 날아 초코렛의 가격을 올린다.
\item 매출이 늘다/뛰다/줄다 \\
  아이팟의 등장부터 매출이 뛰어넘는다.
\item 효과가 있다/없다 \\
  시험에게 공부하는 시간보다 잠자는 시간이 더 큰 효과가 있다. 
\item 대박(이) 나다 \\
  룰렛에서 작은 내기로 대박이 났다.
\end{enumerate}

\section{사자성어}
\begin{enumerate}
\item 先見之明 (선견지명) \\
  선견지명이 없어서 1941년 6월 22일에 히틀러가 소비에트 연방을 공격했다.
\item 錦上添花 (금상첨화) \\
  논문이 학회에 부친 다음에 상도 받을 때 금상첨화였다.
\end{enumerate}

\section{모르는 다어들}
\begin{enumerate}
  \item \ldots
\end{enumerate}
\section{쓰기}
\doublespacing
\subsubsection*{상품을 하나 골라서 어떻게 개선(improve)시킬 수 있을지 써 보세요.}

티비를 생각하면, 많이 변해왔다. 엿날에 색깔도 없고 냉장고만큼 컸는데, 이제 얇고 커다란 화면을 보면 많이 올라왔다.

하지만, 아직도 티비 방송 똑같다. 지상파의 변화 별로 없고 좋아하는 프로그랩이 정해진 시간만에 나온다. 티비를 처음 사고 집에 넣으면, 캐이블이나 다른 연결될 수 있는 기계가 살 때까지, 거의 아무것도 볼 수 없다.

내 상상에는, 티비가 와이파이를 찾고 인터넷으로 재미있는 프로그램을 찾아줄 수 있다. 훌루나 넷플릭스나 유튜브나 다른 웹사이트에서 티비쇼나 다른 동영상이 많기 때문에 티비에서 못 볼 이유가 없다.  그래서 티비를 처음 킬 때부터 볼 수 있는 티비쇼를 보여줄 수 있고, 아무 에피소드를 볼 수 있다.

원래 티비처럼 동영상이 계속 나오는 기능도 있을 수 있지만, 더 똑똑하게 할 수 있다.  리모콘에세 ``좋다"와 ``싫다"는 버튼이 있을 수 있다.  ``좋다"를 누르면, 나중에 비슷한 프로그랩을 찾아줄 수 있고 ``싫다"를 누르면, 지금 보고 있는 쇼를 그만하고 완전히 다른 것을 보여줄 수 있다.
\end{document}
