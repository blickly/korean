\documentclass[11pt]{article}
\usepackage{kotex}
\usepackage{setspace}
\usepackage{fullpage}
\renewcommand \thesection{\Roman{section}}

\title{한국어 102: 숙제 \#4}
\author{벤}
\date{2010년 3월 14일}

\begin{document}
\maketitle
\thispagestyle{empty}
\pagestyle{empty}

\section{표현}
\begin{enumerate}
\item 성공을 거두다
\item 의견을 내다/제시하다
\item 의견을/뜻을/요구를/요청을 받아들이다/따르다/감안하다/반영하다
\begin{enumerate}
\item ...
\item ...
\end{enumerate}
\item (불)가능한 일로 여기다
\item 맛을 내다
\item 가격을 올리다/내리다
\item 매출이 늘다/뛰다/줄다
\item 효과가 있다/없다
\item 대박(이) 나다
\end{enumerate}

\section{사자성어}
\begin{enumerate}
\item 선견지명 (          )
\item 금상첨화 (          )
\end{enumerate}

\section{모르는 다어들}
\begin{enumerate}
  \item \ldots
\end{enumerate}
\section{쓰기}
\doublespacing
\subsubsection*{상품을 하나 골라서 어떻게 개선(improve)시킬 수 있을지 써 보세요.}

티비를 생각하면, 많이 변해왔다. 엿날에 색깔도 없고 냉장고만큼 컸는데, 이제 얇고 커다란 화면을 보면 많이 올라왔다.

하지만, 아직도 티비 방송 똑같다. 지상파의 변화 별로 없고 좋아하는 프로그랩이 정해진 시간만에 나온다. 티비를 처음 사고 집에 넣으면, 캐이블이나 다른 연결될 수 있는 기계가 살 때까지, 거의 아무것도 볼 수 없다.

내 상상에는, 티비가 와이파이를 찾고 인터넷으로 재미있는 프로그램을 찾아줄 수 있다. 훌루나 넷플릭스나 유튜브나 다른 웹사이트에서 티비쇼나 다른 동영상이 많기 때문에 티비에서 못 볼 이유가 없다.  그래서 티비를 처음 킬 때부터 볼 수 있는 티비쇼를 보여줄 수 있고, 아무 에피소드를 볼 수 있다.

원래 티비처럼 동영상이 계속 나오는 기능도 있을 수 있지만, 더 똑똑하게 할 수 있다.  리모콘에세 ``좋다"와 ``싫다"는 버튼이 있을 수 있다.  ``좋다"를 누르면, 나중에 비슷한 프로그랩을 찾아줄 수 있고 ``싫다"를 누르면, 지금 보고 있는 쇼를 그만하고 완전히 다른 것을 보여줄 수 있다.
\end{document}
