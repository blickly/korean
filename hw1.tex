\documentclass{article}
\usepackage{kotex}
\renewcommand \thesection{\Roman{section}}

\title{한국어 102: 숙제 \#1}
\author{벤}
\date{2010년 2월 1일}

\begin{document}
\maketitle
\thispagestyle{empty}
\pagestyle{empty}

\section{표현}
\begin{enumerate}
  \item 새해가 밝다 \\
키리바시에서 새해가 재일 먼저 밝는다.

  \item 미래로 뻗어 가다 \\
크린턴 전대통령이 '미래로 뻗어 갈 다리'를 놓았다.

  \item 기회를 만들다 \\
기회를 기다리지 말고 자기의 기회를 만들어야 한다고 들었다.

  \item 길을 열다 \\
김대중 전대통령이 북한의 길을 열었다.

  \item 원조를 주다/받다 \\
원조를 줄 생각이면, 아이티 귀표가 없는 원조는 더 유용하대.

  \item 성과를 이루다 \\
성과를 이루려면, 매일 연습해야 된다.

  \item 자신감을 얻다 \\
음치라도, 자신감 얻고 노래하면, 훨씬 더 재미있다.

  \item 초석을 다지다 \\
으리으리한 성에도,  초석을 다져야 된다.

  \item 일류로 도약하다 \\
``캐리"가 나올때, 스티번 킹 작가가 일류로 도약했다.

  \item 복을 받다 \\
복을 받은 우리 할아버지가 제2차 세계 대전을 피하게 되었다.
\end{enumerate}

\section{사자성어}
\begin{enumerate}
  \item 일로영일 \\
재고를 파는 친구가 일로영일로 많이 얻었지만, 내가 해봤을때 거의 다 잃어버렸다.

  \item 송구영신 \\
12월 31일에 송구영신하는 파티를 다녔다.

  \item 근하신년 \\
이번 새해에, 생각만큼 근하신년이라는 카드를 못 받았다.
\end{enumerate}

\section{모르는 단어들}
\begin{enumerate}
  \item 뻗다 - reach, stretch
  \item 정상 - summit
  \item 의장 - chairperson
  \item 주최 - host
  \item 숙원 - 오랫동안 원했던 소원
  \item 원자력 - nuclear power
  \item 발전소 - power plant
  \item 수출 - export
  \item 원조 - aid
  \item 이루다 - accomplish
\end{enumerate}

\section{쓰기}
자기가 신년사를 해야 하면, 먼저 여러 사람들이 하던 걸 배우고 할 거다.  미국에서 꼭 신년사가 아니어도, 대통령 연두 교서가 재일 비슷하다.  수요일에 오바마가 연두 교사를 했고, 나는  비슷하게 하려고 노력했을 거다.  오바마는 연설 잘 하는 편이라서, 오바마보다 잘 할 수 있을 지 모르겠다.  한 면 밖에 안 바꿀 거다.  수요일 교서는 좀 너무 길었다. 내 생각에, 보는 사람이 70분 계속 집중하기 어려운 거다.

미국 연두 교서와 달리, 이명박이 하는 신년사에 정책이 없었다.  거의 다 한국의 찬사였다.  농담도 없었고, 내용에 새로운 생각이나 의견이 없다. 내 생각에 한국에 살면서 신년사를 들었으면, 아무 새로운 것도 안 배웠을 거다.  그래서 내용도 중분히 신선하지 않으면, 아무도 안 볼 것 같다.  근대 좋은 점도 있었다.  중분히 짧았다.  

자기가 신년사를 해야 하면, 오바마와 이명박의 좋은 점을 합치려고 할 거다.  오바마처럼 농담도 놓고 자기의 새로운 생각을 놓으려고 할 거고, 이명박처럼 시청자가 집중할 수 있을 만큼 짧게 하려고 할 거다. 내가 30분간에 재일 중요한 부분만 넣으려고 노력할 거다.  그래서 생각보다 길게 45분간으로 나와도, 24의 에피소드보다 길지 않다. (광고때문에, 에피소드가 한시간을 걸려도, 집중하는 시간 더 적다.)

\end{document}
