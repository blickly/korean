\documentclass[12pt]{article}
\usepackage{kotex}
\usepackage{setspace}
%\usepackage{fullpage}

\title{한국어 102: 쓰기 숙제}
\author{벤}
\date{2010년 2월 22일}

\begin{document}
\maketitle
\thispagestyle{empty}
\pagestyle{empty}
\doublespacing

금요일에 우리반 ``국가 대표"라는 영화를 봤다.  나는 원래 현실적인 영화를 좋아해서, ``국가 대표"가 실화를 토대로 만들었다고 들었을 때 좋아할 줄 알았다. 하지만 현실적인 영화가 아니었다.

너한테, 미국에 대한 비현실적인 점들이 제일 잘 보였다. 그리고 믿을 수 없는 부분이 자주 나올 때, 얘기가 유창할 수 없다.

나의 입양 된 친구하고 비교하면 주인공과 여동생이  하나도 비슷하지 않았다.  7살이면 아주 어려서 그만큼 깊은 모국 인상이 남을 확률이 아주 낮다. 그리고 ``한국어 베우려고 한글학교와 한국교회에 다니면 된다"는 얘기를 너무 억지로 만든 티 났다.  현실에서 입양 애들이 모국에 대해서 궁금한 경우가 많고 부모님에게 버림 받았던 느끼는 경우도 있기는 한다. 하지만 나라에게 버림받는 느낌이 별로 없는 것 같다. (근데, ``나라에게 버림바다"는 게 무슨 뜻인가? 정치가 어떤 한국가족한테 입양 안 시켜서 실망한 걸까? 그래서 입양한 가족보다 하나도 모르는 한국가족이 더 좋다는 말이다. 하지만 어느아이가 자기 부모님보다 모르는 부모님 원하면, 부모님을 싫어하는 뜻이다.  사춘기 때 그런 감정을 느껴도 몇년 다음에 곧 없어진다.  그래서 27살 된 주인공한테 안 어울렸다.) 그래서 주인공과 여동생 둘다가 계속 ``한국에게 버림받았다"는 걸 믿을 수 없었다 . 아마 얘기를 더 슬프게 만들었는데 부자연스러워 느꼈다.

다른 믿을 수 없는 점은 미국팀의 국가주의적인 행동이었다.  현실에서 누가 미국팀처럼 다른 팀을 자극하면 순위에 올라갈 수 없을 만큼 실격이 될 거다.  하지만 처음 만날 때부터 미국팀이 한국팀을 놀리는 게 말이 안 됐어도, 2002년 올림픽 주최국을 선택하는 장면이 제일 부자연스러웠다. 왜 솔트 레이크 시티에 대해서 신경 쓴 지 모르겠고, 왜 올림픽이 솔트 레이크 시티로 오는 게 환호한 일인 지도 모르겠다. 2002년 올림픽 위치와 1998년 선수단이 전혀 과계 없는 것 같았다. 그리고 만약에 관계가 있었더라도 (홈 필드 어드밴티지, 등) 환호할 만큼 큰 일이 아니다. 내 생각으로 모국에서 하고 싶어한 선수가 있어도 다른 나라 가보고 싶어한 선수도 있을 거다.  전체적으로 미국팀의 과장된 국가주의 때문에 캐릭터와 상황이 부자연스러워 느꼈다.

\end{document}
