\documentclass[11pt]{article}
\usepackage{kotex}
\usepackage{setspace}
\usepackage{fullpage}
\renewcommand \thesection{\Roman{section}}

\title{한국어 102: 숙제 \#3}
\author{벤}
\date{2010년 2월 24일}

\begin{document}
\maketitle
\thispagestyle{empty}
\pagestyle{empty}

\section{표현}
\begin{enumerate}
  \item 막을 올리다/내리다 \\
    \ldots
  \item 기대를 하다/받다 \\
    \ldots
  \item 활약이 눈부시다 \\
    \ldots
  \item N에 힘(을) 입다 \\
    \ldots
  \item 성적을 올리다/거두다 \\
    \ldots
  \item 메달을 따다/노리다/차지하다 \\
    \ldots
  \item 눈총을 주다/받다 \\
    \ldots
  \item 파란/빨간 불을 켜다 \\
    \ldots
  \item 간판 \\
    \ldots
  \item 사기가 오르다/떨어지다 \\
    \ldots
\end{enumerate}

\section{사자성어}
\begin{enumerate}
  \item 고진감래 \\
    \ldots
  \item 어부지리 \\
    \ldots
\end{enumerate}

\section{모르는 다어들}
\begin{enumerate}
  \item 역대 - 여태까지
  \item 연속 - continuity
  \item 빙속 - 스피드 스케이탕
  \item 종합 - synthesis
  \item 수직 - vertical, perpendicular
  \item 상승하다 - 올라가다
  \item 눈부시다 - 밝고 아름답다
  \item 활약 - 큰 활동
  \item 마저 - 까지도
  \item 제치다 - (상대방보다) 우위에 서다 
\end{enumerate}

\section{쓰기}
\doublespacing
1500미터 쇼트트랙에서 아폴로 오노가 어부지리로 은메달을 땄다는 사람이 많다.  오노가 솜씨보다 운으로 이겼다는 말이다. 한국 선수들의 충돌 때문에 4위부터 2위로 가는 마지막 5초 때 오노의 운이 좋았지만, 그전 1위에서 4위로 덜어지는 5초 때 운이 나빴다. 

내 생각에 사람들이 원하는 결과가 자연히 더 가까워 느낀다. 그래서 스포츠에서 까가울 때 둘다 자기가 이겼다고 생각할 수 있다.  "심판이 틀렸어, 내가 졌다"는 선수를 상상될까? 매번 정반대로 한다.

오노과 한국 선수들 다 솜씨가 많는 게 분명하다. 물론 오노의 은메달이 운 때문이기는 하는데, 이정수의 금메달도 운 때문이다. 운이 없었으면 경기하기 전에 이길 선수를 알 수 있고 아무 재미도 없을 거다. 운 떡분에 올림픽을 보기가 재미있다.

2002년 솔트 래이크 시티 올림픽 1000미터 쇼트트팩에서 8위보다 잘 한 적이 없었던 스티븐 브래드베리가 금메달을 차지했다. 더 빨리 타는 선수들 네명이 다 큰 충돌이 되기 때문이었다. 어부지리인가? 물론. 근데 아마 이런 믿을 수 없는 순간이 제일 중요하다.

\end{document}
