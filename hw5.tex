\documentclass[11pt]{article}
\usepackage{kotex}
\usepackage{setspace}
\usepackage{fullpage}
\renewcommand \thesection{\Roman{section}}

\title{한국어 102: 숙제 \#5}
\author{벤}
\date{2010년 4월 7일}

\begin{document}
\maketitle
\thispagestyle{empty}
\pagestyle{empty}

\section{표현}
\begin{enumerate}
\item 필요에 의해(서) \\
  매주 마겟에 가기보다 필요에 의해서 조금씩 장보기 더 좇대.
\item 마음이 쓰이다/마음을 쓰다 \\
  작년 여름에 우리 바둑이 도망갔을 삼일 동안 내 마음이 쓰였다.
\item 정성을 다하다 \\
  성공 될 때까지 우리 아빠가 직장에서 정성을 다했다.
\item 목청을 돋우다 \\
  아빠가 원래 조용하기 때문에 목청을 돋웠을 때 큰일 인 줄 알았다.
\item 생각이 나다 \\
  아무 생각이 없어도 손가락부터 아무거나 타이프해버리면, 쓰다가 생각이 난다.
\item 눈에 아른거리다 \\
  처음에 가본 야구 시합이 눈에 아른거려 생생하게 기억이 난다.
\item 아니나 다를까 \\
  아니나 다를까 철수가 다섯번째 숙제도 안 했다.
\item 성에 차다 \\
  아무리 잘 해도 금메달을 딸 때까지 성에 차지 않았다.
\item 눈을 멀게 하다 \\
  자기가 쓴 글을 읽을 때, 눈을 멀게 한다.
\item 마음이/마음을 상하다 \\
  아내가 셋서방과 도망갔을 때 철수의 마음이 상했다.
\end{enumerate}

\section{사자성어}
\begin{enumerate}
\item 主客顚倒 (주객전도) \\
\item 愛之重之 (애지중지) \\
\end{enumerate}

\section{모르는 다어들}
\begin{enumerate}
  \item 
  \item 
  \item 
  \item 
  \item 
  \item 
  \item 
  \item 
  \item 
  \item 
\end{enumerate}
\section{쓰기}
\doublespacing
\subsubsection*{``무소유"라는 제목으로 글을 써 보세요.}

\end{document}
