\documentclass[11pt]{article}
\usepackage{kotex}
\usepackage{setspace}
\usepackage{fullpage}
\renewcommand \thesection{\Roman{section}}

\title{한국어 102: 숙제 \#5}
\author{벤}
\date{2010년 4월 7일}

\begin{document}
\maketitle
\thispagestyle{empty}
\pagestyle{empty}

\section{표현}
\begin{enumerate}
\item 필요에 의해(서) \\
  매주 마겟에 가기보다 필요에 의해서 조금씩 장보기 더 좇대.
\item 마음이 쓰이다/마음을 쓰다 \\
  작년 여름에 우리 바둑이 도망갔을 삼일 동안 내 마음이 쓰였다.
\item 정성을 다하다 \\
  성공 될 때까지 우리 아빠가 직장에서 정성을 다했다.
\item 목청을 돋우다 \\
  아빠가 원래 조용하기 때문에 목청을 돋웠을 때 큰일 인 줄 알았다.
\item 생각이 나다 \\
  아무 생각이 없어도 손가락부터 아무거나 타이프해버리면, 쓰다가 생각이 난다.
\item 눈에 아른거리다 \\
  처음에 가본 야구 시합이 눈에 아른거려 생생하게 기억이 난다.
\item 아니나 다를까 \\
  아니나 다를까 철수가 다섯번째 숙제도 안 했다.
\item 성에 차다 \\
  아무리 잘 해도 금메달을 딸 때까지 성에 차지 않았다.
\item 눈을 멀게 하다 \\
  자기가 쓴 글을 읽을 때, 눈을 멀게 한다.
\item 마음이/마음을 상하다 \\
  아내가 셋서방과 도망갔을 때 철수의 마음이 상했다.
\end{enumerate}

\section{사자성어}
\begin{enumerate}
\item 主客顚倒 (주객전도) \\
  성적을 위해서 공부하면, 주객전도 된다.
\item 愛之重之 (애지중지) \\
  자기의 자녀보다 엄마가 아빠한데서 받던 반지를 애지중지한다.
\end{enumerate}

\section{모르는 다어들}
\begin{enumerate}
  \item 물레 - spinning wheel
  \item 교도소 - 감옥
  \item 깡통 - can
  \item 허름하다 - 싸거나 좀 모자라다
  \item 요포 - 담요
  \item 평판 - reputation
  \item 참석하다 - attend, participate
  \item 도중 - 가는 길
  \item 소지품 - 가지고 있는 물건
  \item 분수 - one's place
\end{enumerate}
\section{쓰기}
\doublespacing
\subsubsection*{``무소유"라는 제목으로 글을 써 보세요.}

소유가 없는 세상을 상상할 수 있지만, 법정스님만큼 아름다운 상상이 이나다.

이 세상에서 일하는 사람의 두 가지가 있다.  일을 좋아해서 일하는 마음도 있고
일을 해야할 줄 알아서 일하는 마음도 있다.  내 생각에는 일을 싫어해도 해야할 줄
아는 사람이 소유 때문에 일한다.  자기나 사랑하는 가족이 물건을 가짐을 위해서
열심히 일한 사람이 많은 것 같다.  그리고 소유가 없는 세상에서 다들 살 수 있을
만큼보다  일하지 않을 거다.

다들 재미있는 일이나 기분이 좋은 일만 하면, 큰 회사가 될 수 없을 것 같다.
그리고 큰 회사가 없이 여러가지의 고급 기술 물건을 만들 수 없을 것 같다.  새로운
티비, 자동차, 휴대폰 및 컴퓨터를 안 만들면, 다 점점 없어질 거다.


그래서 내 상상에서 좋은 점도 있고 나쁜 점도 있다.  고급 기술이 없어져도 사람들
서로에게 더 베려하는 것은 좋다고 할 수 있다.  하지만, 나에게 이 세상에 살기 더
좋다.

\end{document}
